\section{Theoretical Motivation}
Since the lifetimes of particles in the SM cover an enormous range of magnitude, a variety of particles in the SM are long-lived particles. From the aspect of experiment, the signature of long-lived particles in the SM are well-understood and can be easily removed from signatures searching for new physics. While some similar mechanism  shared by long-lived particles in both SM and BSM. 

In SM, there are several ways to prolong the lifetimes of particles. The first way is off-shell mediator. In the standard model, quark flavor is conserved by all but weak interaction while the decay is highly off-shell due to the large mass of $Z/W$. A example is the decay of charged pion $\pi^{\pm}$, made up with up and down quark, it decays via weak interaction and $m_{W^{\pm}} >> m_{\pi^{\pm}}$, and the decay width scales as 
\begin{equation}
   \Gamma_{\pi^{\pm}} \sim g_{W}^4 (\frac{m_{\pi}}{m_W})^{4} m_{\pi}
\end{equation}

\textbf{Need check}. A second way is nearly mass degeneration which lead to suppressed final-state phase space. In neutron decay $ n^0 \rightarrow p^+ + e^- + \bar{\nu_{e}}$, the mass split between neutron and proton is tiny  due to the isospin symmetry which broken by the masses of quarks and electromagnetic interaction. The neutron decay then also suppressed by off-shell $W^-$. And a third way is small coupling,  a dummy example in SM is flavour-changing neutral currents $\mu \rightarrow e \gamma $ where the coupling is suppressed by tiny neutrino Yukawa couplings \textbf{check with Stuart}. 

With a survey of BSM extensions, LLPs are commonly predicted in a variety of theories that address fundamental mysteries in particle physics such as naturalness, baryogenesis, non-zero neutrino masses and dark matter.

For naturalness problem, Supersymmetry (SUSY) is a well-known solution of the hierarchy problem of higgs mass. Within Minimal Supersymmetric Standard Model (MSSM), Anomaly-mediated supersymmetry breaking (AMSB) predicts a particle mass spectrum in which a small mass splitting exist. The mass splitting can be small as several MeV which compress final-state phase space and allow  macroscopic lifetime. Another case is  Gauge Mediated SUSY breaking(GMSB) with the scale of SUSY breaking $\sqrt{F}$, where the gravitino is the lightest supersymmetric particle (LSP) whose mass scales as 
\begin{equation}
    m_{3/2} \sim \frac{F}{M_{pl}}
\end{equation}
and the coupling of the next-to-lightest supersymmetric particle (NLSP) to LSP scales as $1/F$. There is a range of $F$ where the coupling is sufficiently small and NLSP has macroscopic lifetime. For non-minimal SUSY model, R-parity violating (RPV) interaction is considered given null results for LPC SUSY searches \textbf{[cite]}. Low-energy results have constrained flavor violation interaction significantly \textbf{[cite]} and the LSP which decays into SM particles through RPV interaction should be long-lived particle. 

\iffalse
In the QCD sector, another naturalness concern is the Strong CP problem \textbf{[cite]}, the gauge invariance allows CP-odd term: 
\begin{equation}
    \mathcal{L}_{\theta} = \theta \frac{g^{2}_{S}}{32\pi^{2}}G^{a}_{\mu\nu}\tilde{G}^{\mu\nu}_{a}
\end{equation}
while $\theta$ is a dimensionless parameter. The actual value of $\theta$ is extremely close to zero, $|\theta|< 10^{-10}$, with measurements of nucleon electric dipole moments. By introducing an axion which is a Goldstone boson of a QCD-anomalous Peccei-Quinn (PQ) symmetry \textbf{[cite]}, small $\theta$ value can be explained.
\fi

Baryongenesis is another puzzle which is important for the understanding of early-age universe. Evidences indicate that the observable universe originates from a compressed dense state that filled with a hot primordial plasma and still expanding. If assuming the amounts of matter and anti-matter were exactly the same during that epoch, they would have annihilated as the universe cooled down and no baryons would be left to form galaxies. Hence, the presence of baryons indicates there is asymmetry between matter and anti-matter. The magnitude of \textit{baryon asymmetry of the universe} (BAU) can be estimated by cosmological measurement of baryon to photon ratio, or equivalently the ratio $Y_B = (n_B-n_{\barB})/s$ where $n_B-n_{\barB}$ is the comoving baryon density and $s$ is the entropy density. The value of $Y_B$ can be extracted from both the Cosmic Microwave Background (CMB) anisotropy power spectrum and the abundance of light elements in the intergalactic medium. 